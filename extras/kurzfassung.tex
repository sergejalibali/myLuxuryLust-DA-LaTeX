%
% neue Seite
%
\newpage

%
% Ueberschrift Kurzfassung
%
\section*{Kurzfassung}
Das Ziel ist eine professionelle dynamische Webseite und eine Android Applikation zu
erstellen, damit für den Benutzer die weltweit neusten ''Trends'' jederzeit verfügbar sind.
Wichtig ist uns die Benutzerfreundlichkeit, darum werden wir ein Design erstellen, das die
Navigation leicht macht. Damit die Besucher uns nicht vergessen, wird es die Möglichkeit
geben für ein Newsletter zu abonnieren und auch einen Überblick von den wöchentlichen
Inhalt der Seite erhält.
Jedes Artikel, das auf der Webseite oder auf der App ist, kann über soziale Medien
veröffentlicht werden, was auch eine Art von Werbung sein wird, damit die Webseite oder
App populärer werden und somit Anzahl an Besucher gewinnen. Auf dieser Art und Weise
hat man auch die Möglichkeit mit Affiliate-Systemen Profit zu machen.
SEO-Optimierung ist auch ein wichtiges Thema worüber wir denken müssen, weil je besser
die Webseite strukturiert und nach den neusten Standards (HTML5) gebaut ist, desto höher
wird es bei den Suchmaschinen wie Google, Yahoo, Bing ankommen, was auf die
Vertrauenswürdigkeit der Webseite weist.
Für die Testung der Webseite und der Applikation werden wir einen Musterinhalt einfügen,
damit wir zeigen wie diese funktionieren, aber das Management der Seite und App werden
selbst von den Auftraggebern gemacht.


%\color{red} 
%Die Kurzfassung fasst die Arbeit in max. 200 Worten zusammen. 
%\begin{itemize}
 %\item Was ist das Problem / die Aufgabenstellung / Fragestellung gewesen?
 %\item Was ist das Ziel der Diplomarbeit?
 %\item Theoretischer Hintergrund
 %\item Methodik(en)
 %\item Was ist das (Kern-)Ergebnis?
%\end{itemize}
%Die Kurzfassung ersetzt gewisserma{\ss}en das \"Uberfliegen des eigentlichen Textes!
%Die Kurzfassung soll einen \"Uberblick \"uber die Arbeit geben, sowie den \glqq{}roten Faden\grqq{} und die wichtigsten Details f\"ur den Leser liefern. Sie muss informativ sein, unabh\"angig ob sie alleine oder zusammen mit der Arbeit gelesen wird.
%Eine gute Kurzfassung hat selten mehr als 100 - 200 Worte und fasst kurz und pr\"agnant die Thematik, das Ziel der Arbeit, die verwendeten Methoden und (Kern-)Ergebnisse bzw. Erkenntnisse zusammen. Die Kurzfassung ist der zuletzt erstellte Teil der Arbeit.

\color{black} 

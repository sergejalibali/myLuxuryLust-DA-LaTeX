%
% neue Seite
%
\newpage

%
% Ueberschrift Kurzfassung
%
\section*{Kurzfassung}
Text\\
TextTextTextTextTextTextTextTextTextTextTextTextText TextTextTextTextTextTextTextTextTextTextTextTextText TextTextTextTextTextTextTextTextTextTextTextTextText TextTextTextTextTextTextTextTextTextTextTextTextText TextTextTextTextTextTextTextTextTextTextTextTextText

\color{red} 
Die Kurzfassung fasst die Arbeit in max. 200 Worten zusammen. 
\begin{itemize}
 \item Was ist das Problem / die Aufgabenstellung / Fragestellung gewesen?
 \item Was ist das Ziel der Diplomarbeit?
 \item Theoretischer Hintergrund
 \item Methodik(en)
 \item Was ist das (Kern-)Ergebnis?
\end{itemize}
Die Kurzfassung ersetzt gewisserma{\ss}en das \"Uberfliegen des eigentlichen Textes!
Die Kurzfassung soll einen \"Uberblick \"uber die Arbeit geben, sowie den \glqq{}roten Faden\grqq{} und die wichtigsten Details f\"ur den Leser liefern. Sie muss informativ sein, unabh\"angig ob sie alleine oder zusammen mit der Arbeit gelesen wird.
Eine gute Kurzfassung hat selten mehr als 100 - 200 Worte und fasst kurz und pr\"agnant die Thematik, das Ziel der Arbeit, die verwendeten Methoden und (Kern-)Ergebnisse bzw. Erkenntnisse zusammen. Die Kurzfassung ist der zuletzt erstellte Teil der Arbeit.

\color{black} 
